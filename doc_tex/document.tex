\documentclass[a4paper,11pt]{article}
\usepackage[utf8]{inputenc}
\usepackage{amsmath}
\usepackage{amsfonts}
\usepackage{amssymb}
\usepackage{graphicx}
\usepackage{enumitem}
\usepackage{listings}
\usepackage{xcolor} % kvuli barvickam v pythonim textu 
\usepackage{xspace}
\usepackage{physics}
\usepackage[margin=2.5cm]{geometry} % nastavuje sirku okraju stranky

\author{Anna Juranova}
\title{Ferrers Potential in galpy}

% Dots following numbers of section:
\renewcommand{\thesection}{\arabic{section}.}
\renewcommand{\thesubsection}{\thesection\arabic{subsection}.}
% Only one letter wide space between the number and the title of the chapter:
\makeatletter \def\@seccntformat#1{\csname the#1\endcsname\hspace{1ex}} \makeatother

\lstdefinestyle{customc}{
	%belowcaptionskip=1\baselineskip,
	breaklines=true,
	%frame=L,
	%xleftmargin=\parindent,
	language=Python,
	showstringspaces=false,
	keepspaces=true,
	basicstyle=\footnotesize\ttfamily,
	keywordstyle=\bfseries\color{green!40!black},
	commentstyle=\itshape\color{purple!40!black},
	%breakatwhitespace=true,
	identifierstyle=\color{blue},
	stringstyle=\color{orange},
}

\lstset{escapechar=@,style=customc}

%%%%%%%%%%%%%%%%%%%%%%%%%%%%%%%%%%%%%%%%%%%%%%%%%%%%%%%%%%%%%%%%%%%%%%%%%%%%%%%%%%%%%%%%%%%%%%%%%%%%%%%%%%%%%%%%%%%%%%%%
\begin{document}
	
	\begin{center}
		\huge{Ferrers Potential in galpy}\\
	\end{center}

	\section{Variables and Units}
%	VARIABLES AND UNITS	
           $ \mathrm{amp} $ -- Amplitude to be applied to the potential (default: 1); can be a Quantity with units of mass or $ G \times \mathrm{mass} $\\
           $ a $ -- Scale radius (can be Quantity)\\
           $ n $ -- Power of Ferrers density ($ n > 0 $), not necessarilly integer, calculations generally fail for $ n > 4 $\\
           $ b $ -- y-to-x axis ratio of the density, usually $ b < a $\\
           $ c $ -- z-to-x axis ratio of the density, usually $ c < b $\\
           $ \Omega_b $ -- pattern speed of the bar, $ \vec{\Omega}_b = \hat{e}_z\,\Omega_b $ - not included in the code yet!\\
           $ \mathrm{zvec} $ -- If set, a unit vector that corresponds to the z axis\\
           $ \mathrm{pa} $ -- If set, the position angle of the x axis (rad or Quantity)\\
           $ \mathrm{glorder} $ -- If set, compute the relevant force and potential integrals with Gaussian quadrature of this order\\
           normalize -- if True, normalize such that $ v_c(1.,0.)=1. $, or, if given as a number, such that the force is this fraction of the force necessary to make $ v_c(1.,0.)=1. $.\\
           $ r_0 $, $ v_0 $ -- Distance and velocity scales for translation into internal units (default from configuration file)\\
           Variables $ \alpha $	appearing in the code in form $ \_\alpha2 $ carry the value of $ \alpha^2 $ initialized in the begining of the class \_\_init\_\_ function.

	\section{Functions}
	(... those I have written so far, all for time-independent potential in a frame of reference which is aligned with axes of the potential)
%	FUNCTIONS	
		 
		\subsection{Evaluate} % EVALUATE
			Purpose: Evaluation of the potential as a function of (x,y,z) in the aligned coordinate frame \\
   		\begin{equation}
   		\Phi(\vec{x}) = \frac{-\mathrm{amp}\,b\,c}{4(n+1)}\,\Phi'
   		\end{equation}	
	   		\lstinputlisting[language=Python, firstline=125, lastline=126]{FerrersPotential.py}
	
		
		\subsection{X Force} % X-FORCE
			Purpose: Evaluation of the x component of the force as a function of (x,y,z) in the aligned coordinate frame \\
   		\begin{equation}
   		F_x = \frac{-\mathrm{amp}\,b\,c}{2}\,\Theta'_1
   		\end{equation}  
			\lstinputlisting[language=Python, firstline=263, lastline=264]{FerrersPotential.py}

		\subsection{Y Force} % Y-FORCE
			Purpose: Evaluation of the y component of the force as a function of (x,y,z) in the aligned coordinate frame \\
   		\begin{equation}
   		F_x = \frac{-\mathrm{amp}\,b\,c}{2}\,\Theta'_2
   		\end{equation}	
	   		\lstinputlisting[language=Python, firstline=272, lastline=273]{FerrersPotential.py}
	
		
		\subsection{Z Force} % Z-FORCE
			Purpose: Evaluation of the z component of the force as a function of (x,y,z) in the aligned coordinate frame \\
   		\begin{equation}
   		F_x = \frac{-\mathrm{amp}\,b\,c}{2}\,\Theta'_3
   		\end{equation}
	   		\lstinputlisting[language=Python, firstline=281, lastline=282]{FerrersPotential.py}	
	
		
		\subsection{Density} % DENSITY
			Purpose: Evaluation of the density as a function of (x,y,z) in the aligned coordinate frame from cylindrical coordinates given as input\\
		\begin{equation}
		\rho(x,y,z) = \frac{\mathrm{amp}}{4\,\pi\,a^3}\,(1-(m/a)^2)^n,
		\end{equation}
		where
		\begin{equation}
		m^2 = x'^2 + \frac{y'^2}{b^2}+\frac{z'^2}{c^2}
		\end{equation}
			\lstinputlisting[language=Python, firstline=434, lastline=442]{FerrersPotential.py}


		% SECOND DERIVATIVE
		\subsection{General Second Derivative}
			Purpose: General 2nd derivative of the potential as a function of (x,y,z) in the aligned coordinate frame \\
		\begin{equation}
		\Phi_{\mathrm{ij}} = -\frac{1}{4}\,b\,c \,\Phi'_{\mathrm{ij}}	
		\end{equation}
			
			\lstinputlisting[language=Python, firstline=425, lastline=426]{FerrersPotential.py}
	
	
		% INTEGRATION FOR POTENTIAL
		\subsection{Integration for Potential}
			Purpose: Evaluation of the z component of the force as a function of (x,y,z) in the aligned coordinate frame \\
		\begin{equation}
		\Phi' = \int_{0}^{\infty} A^{n+1}(\tau)\,\mathrm{d} \tau
		\end{equation}
			\lstinputlisting[language=Python, firstline=484, lastline=490]{FerrersPotential.py}
	
	
		% INTEGRATION FOR FORCE
		\subsection{Integration for Forces}
			Purpose: Evaluation of the z component of the force as a function of (x,y,z) in the aligned coordinate frame \\
		\begin{equation}
		\Theta'_i = \int_0^{\infty} \frac{x_i}{a_i^2 + \tau} A^{n}(\tau)\,\mathrm{d} \tau,
		\end{equation}
		where
		\begin{equation}
		a_1 = a,~ a_2 = ab,~a_3 = ac
		\end{equation}
		
			\lstinputlisting[language=Python, firstline=497, lastline=503]{FerrersPotential.py}
	
		
		% INTEGRATION FOR SECOND DERIVATIVE	
		\subsection{Integration for Second Derivative}
			Purpose: Integral that gives the 2nd derivative of the potential in x,y,z \\
			
			\noindent The derivative is generally $ \pdv{\Phi}{x_i}{x_j} $; for $ i==j $ the integral to be evaluated is: 
			
		\begin{equation}
		\Phi'_{\mathrm{ii}} = \int_{0}^{\infty} 
		 \frac{4\,n\,x_i^2}{(\tau + a_i^2)^2} \frac{\left(1- \sum_{i=1}^{3} \frac{x_i^2}{\tau + a_i^2}\right)^{n-1}}{[(\tau + a^2)](\tau + a^2b^2)(\tau + a^2c^c)]^\frac{1}{2}}
		- \frac{2}{\tau + a_i^2} \frac{\left(1- \sum_{i=1}^{3} \frac{x_i^2}{\tau + a_i^2}\right)^{n}}{[(\tau + a^2)](\tau + a^2b^2)(\tau + a^2c^c)]^\frac{1}{2}} \mathrm{d}\,\tau
		\end{equation}
		
			\noindent In all other cases, the integral has this form:
		\begin{equation}
		\Phi'_{\mathrm{ij}} = \int_{0}^{\infty} 
		\frac{4\,n\,x_i\,x_j}{(\tau + a_i^2)(\tau + a_j^2)} \frac{\left(1- \sum_{i=1}^{3} \frac{x_i^2}{\tau + a_i^2}\right)^{n-1}}{[(\tau + a^2)](\tau + a^2b^2)(\tau + a^2c^c)]^\frac{1}{2}} \mathrm{d}\tau
		\end{equation}	
			
			\lstinputlisting[language=Python, firstline=511, lastline=522]{FerrersPotential.py}
		
	
		% PART OF INTEGRAND USED IN ALL FUNCTIONS
		\subsection{Part of Integrand Used in All Functions}
			Purpose: Returns part of an integrand which is used in other functions so the code was more concise. \\
		\begin{equation}
		A^{\nu}(\tau) = \frac{\left(1- \sum_{i=1}^{3} \frac{x_i^2}{\tau + a_i^2}\right)^{\nu}}{[(\tau + a^2)(\tau + a^2b^2)(\tau + a^2c^2)]^\frac{1}{2}}
		\end{equation}
			\lstinputlisting[language=Python, firstline=530, lastline=531]{FerrersPotential.py}
	
	
	
\end{document}