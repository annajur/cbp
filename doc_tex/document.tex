\documentclass[a4paper,11pt]{article}
\usepackage[utf8]{inputenc}
\usepackage{aas_macros}
\usepackage{amsmath}
\usepackage{amsfonts}
\usepackage{amssymb}
\usepackage{graphicx}
\usepackage{enumitem}
\usepackage{listings}
\usepackage{xcolor} % kvuli barvickam v pythonim textu 
\usepackage{xspace}
\usepackage{physics}
\usepackage[margin=2.5cm]{geometry} % nastavuje sirku okraju stranky

\author{Anna Juranova}
\title{Ferrers Potential in galpy}

% Dots following numbers of section:
\renewcommand{\thesection}{\arabic{section}.}
\renewcommand{\thesubsection}{\thesection\arabic{subsection}.}
% Only one letter wide space between the number and the title of the chapter:
\makeatletter \def\@seccntformat#1{\csname the#1\endcsname\hspace{1ex}} \makeatother

\lstdefinestyle{customc}{
	%belowcaptionskip=1\baselineskip,
	breaklines=true,
	%frame=L,
	%xleftmargin=\parindent,
	language=Python,
	showstringspaces=false,
	keepspaces=true,
	basicstyle=\footnotesize\ttfamily,
	keywordstyle=\bfseries\color{green!40!black},
	commentstyle=\itshape\color{purple!40!black},
	%breakatwhitespace=true,
	identifierstyle=\color{blue},
	stringstyle=\color{orange},
}

\lstset{escapechar=@,style=customc}

%%%%%%%%%%%%%%%%%%%%%%%%%%%%%%%%%%%%%%%%%%%%%%%%%%%%%%%%%%%%%%%%%%%%%%%%%%%%%%%%%%%%%%%%%%%%%%%%%%%%%%%%%%%%%%%%%%%%%%%%
\begin{document}
	
	\begin{center}
		\huge{Ferrers Potential in galpy}\\
	\end{center}

	\section{Variables and Units}
%	VARIABLES AND UNITS	
           $ \mathrm{amp} $ -- Amplitude to be applied to the potential (default: 1); can be a Quantity with units of mass or $ G \times \mathrm{mass} $\\
           $ a $ -- Scale radius (can be Quantity)\\
           $ n $ -- Power of Ferrers density ($ n > 0 $), not necessarilly integer, calculations generally fail for $ n > 4 $\\
           $ b $ -- y-to-x axis ratio of the density, usually $ b < 1 $\\
           $ c $ -- z-to-x axis ratio of the density, usually $ c < b $\\
           $ \Omega_b $ -- pattern speed of the bar, $ \vec{\Omega}_b = \hat{e}_z\,\Omega_b $\\
           $ \mathrm{pa} $ -- If set, the initial position angle of the bar in the xy plane measured from x (rad or Quantity)\\
           normalize -- if True, normalize such that $ v_c(1.,0.)=1. $, or, if given as a number, such that the force is this fraction of the force necessary to make $ v_c(1.,0.)=1. $.\\
           $ r_0 $, $ v_0 $ -- Distance and velocity scales for translation into internal units (default from configuration file)\\
           Variables $ \alpha $	appearing in the code in form $ \_\alpha2 $ carry the value of $ \alpha^2 $ initialized in the begining of the class \_\_init\_\_ function.

	\section{Functions}
	 ...implemented for rotating potential in a frame of reference in which the bar lies aligned with x axis.)
%	FUNCTIONS	
		 
		\subsection{Potential} % POTENTIAL ITSELF
			Purpose: Evaluation of the bar potential as a function of cartesian coordinates in corotating frame of reference. \\
   		\begin{equation}
   		\Phi(\vec{x}) = \frac{-\mathrm{amp}\,b\,c}{4(n+1)}\,\int_{\lambda}^{\infty} A^{n+1}(\tau)\,\mathrm{d} \tau
   		\end{equation}
   			where
		\begin{equation}
		A^{\nu}(\tau) = \frac{\left(1- \sum_{i=1}^{3} \frac{x_i^2}{\tau + a_i^2}\right)^{\nu}}{[(\tau + a^2)(\tau + a^2b^2)(\tau + a^2c^2)]^\frac{1}{2}}
		\end{equation}
   			is part of an integrand which is repeatedly used in following functions. Lower limit for the integral based on definition of the density distribution is given by relation:
 
		\begin{equation}
		\lambda = \begin{cases}
		\text{unique positive solution of} \, m^2(\lambda)=1, &  \text{for}\ m \geq 1\\
		\lambda = 0\,, & \text{for}\ m < 1
		\end{cases} \\
		\end{equation}
		where $ m^2(\lambda) = \sum_{i=1}^{3} \frac{x_i^2}{\lambda + a_i^2} $.	
	
	% they have the units in density \rho_0 wrong in paper one (probably because of the constants before the physical variables obtained somehow??) but still! [\rho] \neq s^{-2} !!
		\subsection{Density} % DENSITY
			Purpose: Evaluation of the density as a function of (x,y,z) in the aligned coordinate frame from cylindrical coordinates given as input\\
		\begin{equation*}
		\rho(m^2) = \begin{cases}
		\frac{\mathrm{amp}}{4\,\pi\,a^3}\,(1-(m/a)^2)^n, & \text{for}\, m < a \\
		0, & \text{for}\, m \geq a
		\end{cases}
		\end{equation*}
			where
		\begin{equation}
		m^2 = x^2 + \frac{y^2}{b^2}+\frac{z^2}{c^2}
		\end{equation}
		
		\subsection{Force} % FORCE
			Evaluation of the x component of the force as a function of (x,y,z) in the aligned coordinate frame, which is used for evaluation of the force in cylindrical coordinates and then in orbit integration; does not take into account bar's rotation or initial position and therefore shall not be used directly.\\
   		\begin{equation}
   		F_i = \frac{-\mathrm{amp}\,b\,c}{2}\, \int_{\lambda}^{\infty} \frac{x_i}{a_i^2 + \tau} A^{n}(\tau)\,\mathrm{d} \tau
   		\end{equation}  
		where
		\begin{equation*}
			a_1 = a,~ a_2 = ab,~a_3 = ac
		\end{equation*}
		
		
		% SECOND DERIVATIVE
		\subsection{General Second Derivative}
			General 2nd derivative of the potential as a function of (x,y,z) in the aligned coordinate frame \\
		\begin{equation}
		\Phi_{\mathrm{ij}} = -\frac{1}{4}\,b\,c \,\Phi'_{\mathrm{ij}}	
		\end{equation}
		
		% INTEGRATION FOR SECOND DERIVATIVE	
		\subsection{Integration for Second Derivative}
			Integral that gives the 2nd derivative of the potential in x,y,z \\
			
			\noindent The derivative is generally $ \pdv{\Phi}{x_i}{x_j} $; for $ i=j $ the integral to be evaluated is: 
			
		\begin{equation}
		\Phi'_{\mathrm{ii}} = \int_{\lambda}^{\infty} 
		 \frac{4\,n\,x_i^2}{(\tau + a_i^2)^2} \,A^{\nu-1}(\tau)
		- \frac{2}{\tau + a_i^2} A^{\nu}(\tau) \mathrm{d}\,\tau
		\end{equation}
		
			\noindent In all other cases, the integral has this form:
		\begin{equation}
		\Phi'_{\mathrm{ij}} = \int_{\lambda}^{\infty} 
		\frac{4\,n\,x_i\,x_j}{(\tau + a_i^2)(\tau + a_j^2)} A^{\nu-1}(\tau) \mathrm{d}\tau
		\end{equation}

	% INERTIAL FRAME	
		\subsection{Second Derivative in Nonrotating FoR}
			Transformation of the second derivative into the frame of reference which is corotating with the potential\\
		\begin{equation}
			\Phi_{x'x'} = \cos[2](\Omega_b t)\Phi_{xx} - 2 \sin(\Omega_b t) \cos(\Omega_b t)\Phi_{xy} + \sin[2](\Omega_b t)\Phi_{yy}\\
		\end{equation}
		
		\begin{equation}
			\Phi_{y'y'} = \cos[2](\Omega_b t)\Phi_{yy} + 2 \sin(\Omega_b t) \cos(\Omega_b t)\Phi_{xy} + \sin[2](\Omega_b t)\Phi_{xx}
		\end{equation}
		
		\begin{equation}
			\Phi_{z'z'} = \Phi_{zz}
		\end{equation}
		
		\begin{equation}
			\Phi_{x'y'} = [\cos[2](\Omega_b t) - \sin[2](\Omega_b t)]\Phi_{xy} + \sin(\Omega_b t) \cos(\Omega_b t)[\Phi_{xx} - \Phi_{yy}]
		\end{equation}
		
		\begin{equation}
			\Phi_{x'z'} = \cos(\Omega_b t)\Phi_{xz} - \sin(\Omega_b t) \Phi_{yz}
		\end{equation}
		
		\begin{equation}
			\Phi_{y'z'} = \cos(\Omega_b t)\Phi_{xz} + \sin(\Omega_b t) \Phi_{yz}
		\end{equation}	

	%MILKY WAY BAR POTENTIAL
	\section{Milky Way Bar Potential}
		
		Values of parameters used for further work are set as the final state values in \cite{MachadoManos:2016}, that is:
		
		\begin{equation*}
			a = 8\,kpc,~
			b = 0.35,~
			c = 0.2375,~
			\mathrm{amp} = 3.3 \times 10^{10}\, M_{\odot},~
			\Omega_b = 10\,km/s/kpc.
		\end{equation*}
	
	
		\nocite{*} % Insert publications even if they are not cited in the poster
		
		\small{\bibliographystyle{unsrt}
			\bibliography{sample}\vspace{0.01in}}
		
	
	
\end{document}